\documentclass[11pt,fullpage]{article}
\usepackage[urlcolor=blue,colorlinks=true]{hyperref}

\oddsidemargin 0.0in
\textwidth 6.5in

\begin{document}

\title{Getting Started with the Genome Analysis Toolkit (GATK)}
\author{Matt Hanna}
\date{16 Mar 2009}
\maketitle

\section{Build Prerequisites}
GATK requires JDK 1.6 and Ant 1.7.1 to compile.

\section{Getting and Building the Source}
GATK is located in the Sting svn repository, and
compiles using a build.xml in the root directory.

Download and build the source as follows:
\begin{verbatim}
  svn co https://svnrepos/Sting/trunk Sting
  cd Sting
  ant
\end{verbatim}

\section{Getting Started}
The core concept behind GATK is the walker, a class that implements the 
three core operations: filtering, mapping, and reducing.

\begin{description}
  \item [filter] reduces the size of the dataset by applying a predicate.  
  \item [map] Applies a function to each individual element in a dataset, 
    effectively 'mapping' it to a new element.
  \item [reduce] Inductively combines the elements of a list.  The base
    case is supplied by the reduceInit() function, and the inductive step
    is performed by the reduce() function.
\end{description}
Users of the GATK will provide a walker to run their analyses.  The engine
will produce a result by first filtering the dataset, running a map operation,
and finally reducing the map operation to a single result.  

\section{Creating a Walker}
To be loaded by GATK, the walker must satisfy the following properties:
\begin{enumerate}
  \item It must be a loose class, not packaged into a jar file.
  \item It must be in the unnamed package (in other words, the source
    should not start with a package declaration).
  \item It must subclass one of the basic walkers in the 
    org.broadinstitute.sting.gatk.walkers package: BasicReadWalker or 
    BasicLociWalker.
  \item It must live in the directory \$STING\_HOME/dist/walkers.
\end{enumerate}

\section{Example}
This walker will print output for each read it sees, eventually computing the
total number of reads by mapping every read to 1 and summing all the 1s to
realize the total number of reads.

\begin{samepage}
Copy the following text into the file \$STING\_HOME/dist/walkers/HelloWalker.java:

\begin{verbatim}
import net.sf.samtools.SAMRecord;

import org.broadinstitute.sting.gatk.LocusContext;
import org.broadinstitute.sting.gatk.walkers.BasicReadWalker;

/**
 * Define a class extending from BasicReadWalker with types
 * <MapType,ReduceType>.  
 */
public class HelloWalker extends BasicReadWalker<Integer,Long> {
    private Long currentRead = 0L;

    // Maps each read to the value 1.
    public Integer map(LocusContext context, SAMRecord read) {
        System.out.printf("Hello read %d%n", ++currentRead );
        return 1; 
    }

    // Provides an initial value for the reduce function.
    public Long reduceInit() { return 0L; }
    
    // Defines how to compute the reduction given a value in the list. 
    public Long reduce(Integer value, Long sum) { 
        return sum + value;
    }
}
\end{verbatim}
\end{samepage}
To compile the walker:
\begin{verbatim}
setenv CLASSPATH $STING_HOME/dist/GenomeAnalysisTK.jar:$STING_HOME/dist/sam-1.0.jar
javac HelloWalker.java
\end{verbatim}
To run the walker:
\begin{verbatim}
mkdir $STING_HOME/dist/walkers
java -Xmx4096m -jar dist/GenomeAnalysisTK.jar \
     -I /broad/1KG/legacy_data/trio/na12878.bam -T Hello \
     -L chr1:10000000-10000100 -l INFO
\end{verbatim}
This command will run the walker across a subsection of chromosome 1, operating on 
reads which align to that subsection.  

\end{document}
